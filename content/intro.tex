\section*{ВСТУП}

\addcontentsline{toc}{section}{ВСТУП}

Науковий прогрес завжди заохочував та схиляв людину до якомога більшого розширення межі її знань та спонукав людину спробувати стати, як казав Рене Декарт, \foreignquote{ukrainian}{володарем та хазяїном природи}. Проте, завдяки зростанню розуміння можливих наслідків наукових досліджень та їх технологічних застосувань для людини та суспільства, ми отримали новий фактор: етика вважається інтегральною частиною наукового прогресу.

Прикладна етика у сучасному світі є невід'ємною частиною багатьох сфер людської діяльності, наприклад, ми можемо говорити про етику в бізнесі, етику в юриспруденції та етику в біотехнологіях і в науці загалом, але так було не завжди. Наука та етика – це поняття, які на протязі всієї історії не завжди були тісно переплетеними і невіддільними одне від одного, як є зараз. В 19-му столітті в академічних колах університетів спостерігався бурхливий розвиток наукових досліджень, проте науковці не були сильно зацікавлені у застосуванні їх на практиці. Вони стверджували про нейтральність науки, тобто про те що знання самі по собі не можуть бути хорошими чи поганими, та не переймалися можливими наслідками своєї діяльності на практиці і не відчували відповідальності за такі наслідки. Зовсім іншою була ситуація з наукою в індустрії, оскільки вона мала там зовсім інші цілі та правила. Там дослідження науковців були власністю приватних організацій, а метою цих досліджень було винайдення нових технологій для отримання прибутку, а отже етичні проблеми вважалися проблемами компаній, а не науковців. Як результат, дискусії про етику в науці були практично відсутніми поміж науковцями як в академічних колах так і в індустрії. Лише починаючи з 50-х років 20-го століття, через значне зростання співпраці між академічними колами та індустрією, на передній план дискусій виходять етичні міркування ведення досліджень \cite{Iaccarino2001-ju}. Відтоді дискусії щодо етики в науці залишаються актуальними і все більше зростає важливість таких обговорень саме зараз, коли через стрімкий розвиток науки і технологій та зростання їх складності для розуміння широкому загалу, негативні наслідки неправильних рішень  можуть можуть нести в собі все більшу небезпеку та ставати все тяжчими для суспільства. А ініціаторами таких дискусій повинні ставати саме науковці, адже хто як не вони мають найдетальніші знання та глибоке розуміння об'єкта їхнього дослідження \cite{Copland2003-xt}\cite{Iaccarino2001-ju}.

Проводячи дослідження, науковець завжди приймає численні рішення щодо обрання методології дослідження, способів отримання та використання інформації чи знання. Прийняття таких рішень повинні розглядатися не лише з суто практичної, а й з етичної сторони питання, коли можливі їх наслідки несуть потенційну або навіть реальну шкоду чи небезпеку. При прийнятті рішень науковцем часто можемо спостерігати феномен виникнення моральної дилеми, тобто ситуації, в якій науковець змушений обирати певне рішення при наявності одного або декількох альтернативних рішень, кожне з яких порушує певні морально-етичні правила \cite{sep-moral-dilemmas}. Лише критична дискусія (науковців між собою та з широким загалом) навколо етичної складової прийняття таких рішень та моральних дилем, з якими неодмінно стикатиметься кожен науковець, на мою думку, може поглибити наше розуміння етичності тих чи інших дій науковця та наблизити нас до істини, що правильно а що ні. І таких дискусій на сьогоднішній день ще є недостатньо. Чи варто заборонити клонування людей? Чи потрібно обмежити дослідження із використанням стовбурових клітин людини і якщо так, то з якими обмеженнями це зробити \cite{lachmann2001stem}? Чи варто встановлювати якісь обмеження і якщо так то які саме щодо генної інженерії рослин \cite{trewavas2001opposition}\cite{flothmann2001maize}? Це лише декілька прикладів моральних дилем у сучасній науці у царині біології. Питання етики в такому контексті є настільки важливим, що навіть має свій окремий термін -- біоетика. Саме для вирішення, або хоча б пошуку консенсусу з таких питань, науковці і всі зацікавлені особи повинні піднімати питання питання етики в науковій роботі. Ретельно обдумуючи та враховуючи можливі наслідки своєї діяльності, рефлексуючи на тему наскільки етичним є та чи інша дія, науковець підкреслює, що відповідальність перед суспільством є інтегральною частиною наукової діяльності.

До обрання даної теми мене підштовхнуло те, що мене особисто, у зв'язку з моєю основною професійною діяльністю (комп'ютерне програмування та інноваційні цифрові технології), надзвичайно цікавить розвиток сучасних комп'ютерних наук та новітніх цифрових технологій. Зокрема великий інтерес в мене викликає й етична сторона питання як і у наукових дослідженнях в цій галузі та розробці нових рішень та технологій, що змінюють життя людей на краще (або ж ні, залежно від точки зору), так і у практичних механізмах впровадження таких рішень і технологій у життя (з точки зору державного управління). Оскільки комп'ютерне програмування є моїм фахом та моє дисертаційне дослідження напряму пов'язане з цифровізацією державного управління, частою темою моїх роздумів є тема етики в сучасній науці, зокрема в галузі комп'ютерних наук та медицини, де швидкість прогресу в останні десятиліття є надзвичайно високою. Особливо цікавими та актуальними можемо виділити проблеми у галузях, де комп'ютери повинні приймати важливі рішення, які можуть впливати або від яких може напряму залежати людське життя. Наприклад, питання етики в самокерованих автомобілях, яке рішення повинен прийняти алгоритм керування автомобілем, якщо виникає неоднозначна ситуація \cite{Nyholm2016}\cite{EthicsSelfDriving}? Якими принципами повинен керуватися штучний інтелект та як він повинен себе поводити, коли і якщо він наближатиметься по рівню розуму до людини? Як повинні поводити себе роботи, коли вони повинні приймати рішення, що стосуються людини та мають етичну складову? \cite{bostrom2018ethics}\cite{sep-ethics-ai}.

В чому ж полягає актуальність дослідження ролі етики у сучасній науці? У теперішньому інформаційному суспільстві світі цифрові технології на основі новітніх наукових досліджень стрімко розвиваються і активно впроваджуються в життя людей, проте наслідки таких впроваджень доволі складно передбачити. Саме для передбачення та уникнення небажаних наслідків нам і потрібний строгий етичний процес в науці. Чи означає це, що керуючись етичними нормами та принципами, ми зможемо уникнути усіх помилок та негативних наслідків? Ні, проте, таким чином, ми, принаймні, можемо зробити все залежне від нас для досягнення такої мети.