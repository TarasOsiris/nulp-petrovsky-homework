Для вирішення проблеми браку ресурсів на місцях (матеріальних, людських та ін.) у ОТГ, автором запропоновано створити новий програмний продукт з широким функціоналом (онлайн сервіс) для пошуку таких ресурсів та вирішення суміжних проблем (далі просто \foreignquote{ukrainian}{cервіс}). Основний функціонал цього сервісу полягатиме у можливості представників місцевої влади взаємодіяти з іншими суб'єктами господарської діяльності(сусідніми ОТГ, приватними організаціями та ін.), що володіють ресурсами та можливостями, які, тим чи іншим чином, можуть бути використані для досягнення поставлених цілей чи виконання поставлених задач. У даному розділі запропоновано загальну структуру такого сервісу та зв'язки між його структурними елементами з виділенням його основних та другорядних функцій. Також описано можливі вектори розвитку сервісу при успішному впровадженні його базової версії. 

Для початку, доцільним є визначення основних стейкхолдерів програмного продукту, тобто основних суб'єктів (індивідів, груп та організацій), які зацікавлені в 