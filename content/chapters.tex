\section[РОЗДІЛ 1. СТАНОВЛЕННЯ ЕТИКИ У СУЧАСНІЙ НАУЦІ ТА АКТУАЛЬНИЙ ПОГЛЯД НА НЕЇ]{СТАНОВЛЕННЯ ЕТИКИ У СУЧАСНІЙ НАУЦІ ТА АКТУАЛЬНИЙ ПОГЛЯД НА НЕЇ}

На протязі багатьох століть, у древні часи питання етики як в науці так і в будь-яких інших сферах діяльності людини визначалось традиціями або релігією та були відносно сталими. Зміну таких відносно сталих морально-етичних цінностей у суспільстві спричиняли переважно поширення нових релігій (поширення християнства у Римській імперії) або сильні соціальні потрясіння, такі як війни, революції та колоніалізм (Французька революція, Російська революція, християнське місіонерство при колонізації нових територій). Кількість активних дискусій щодо питань етики в науці різко зросла лише в середині 20-го століття. Основними причинами такого пожвавлення стало значно краще фінансування наукової діяльності у розвинених державах, все активніша співпраця між академічними колами університетів та науковцями в індустрії. Також важливим фактором стало активне зростання ролі етичної складової не лише у використанні результатів досліджень, а й у методології їх проведення, оскільки все більшу роль став відігравати внесок конкретного дослідження у сфери охорони здоров'я, охорони довкілля, харчування, енергетики тощо. Джерелом етичних проблем та дискусій про них ставали стосунки між публічними та приватними дослідженнями, бо все більше науковців почали отримувати фінансову вигоду від результатів своїх досліджень \cite{Iaccarino2001-ju}. Хоча така співпраця науковців з приватним сектором дуже часто заохочується політиками та є вигідною для економіки, все частіше виникають конфлікти інтересів \cite{cech2001conflicts}. З одної сторони науковець прагне бути неупередженим та дотримуватися морально-етичних правил, а з іншої недотримання таких правил може приносити суттєву матеріальну або нематеріальну вигоду. Врегулювання таких конфліктів інтересів все частіше намагаються здійснювати за допомогою законодавства \cite{Smaglik2000}, проте непрозорість і складність їх виявлення знижує ефективність такої практики. Особливо актуальною тема конфлікту інтересів є в медицині, де є поширеною практика клінічних досліджень, при проведенні яких часто виникають етичні проблеми, що стосуються життя та здоров'я пацієнтів, людської гідності \cite{Leader2003}\cite[208]{FrameworkForAction}.

Якими ж етичними нормами та принципами повинен керуватися сучасний науковець? Для пошуку відповіді на це питання доцільно розглянути до якого спільного знаменника спромоглася звести етичне питання міжнародна наукова спільнота, тобто який консенсус панує на початку 21-го століття. ЮНЕСКО та Міжнародна наукова рада в 1999 році у Будапешті організували міжнародну Світову Конференцію про Науку (World Conference on Science). В ній прийняли участь більш ніж 1800 делегатів від 155 країн-членів ООН, 28 міжурядових організацій, більш ніж 60 міжнародних неурядових організацій, 80 міністрів науки та технологій та численна кількість журналістів \cite[7]{WorldConferenceOnScience}, що свідчить про велике охоплення і значущість такої події. Також підготовка до події тривала декілька років, щоб ретельно підготувати документи для обговорення та зібрати інформацію на вході від численних науковців, урядовців та представників громадськості. Конференція стала відповіддю на дилему у суспільстві: виглядало, що публічна підтримка науки зменшувалася, але в той самий час наукові дослідження та технологічний розвиток ставали все необхіднішими для вирішення найскладніших та найактуальніших питань з якими стикнулося людство. Метою конференції було створення нової системи зобов'язань -- нового суспільного контракту -- при якому науковці зобов'язуються нести відповідальність за ці потребами і уряди країн зобов'язуються, в свою чергу, підтримувати дослідження, що намагаються вирішити ці проблеми \cite[5]{WorldConferenceOnScience}. Як результат, урядовим представникам, науковцям природничих та соціальних наук, представникам неурядових організацій, приватному сектору та іншим організаціям вдалося досягти міцного консенсусу не лише щодо важливості збільшення взаємодії та комунікації між усіма зацікавленими в розвитку науки сторонами, а також щодо того, що способи ведення науки є не менш важливими ніж засоби ведення науки, тобто багато уваги було приділено саме етичній стороні питання при проведенні наукових досліджень. Новим \foreignquote{ukrainian}{соцільним контрактом}, по суті, стало твердження, що для того, щоб заслужити довіру широкого загалу, науковці повинні приділяти активну увагу етичним та соціальним питанням. А отже, для того щоб здобути підтримку суспільства та політиків, наука повинна відповідати потребам реальних людей, поважати індивідуальні права людини та розширювати можливості громад. \cite{doi:10.1126/science.285.5427.529}. Проаналізувавши хід підготовки до конференції, саму подію та офіційні документи, прийняті в результаті обговорень \cite[461]{WorldConferenceOnScience}, бачимо що питання відповідальності науки перед суспільством та питання етики при проведенні досліджень та впровадженні їх результатів є одним з центральних питань, ключову важливість якого визнає міжнародна наукова спільнота. Про це також засвідчує блок доповідей про актуальні питання та проблеми відповідальності науки та етики в науці на самій конференції (I.11 Science, Ethics and Responsibility) \cite[206]{WorldConferenceOnScience}.

Для глибшого розуміння загального сучасного консенсусу щодо етики в науці доцільним є аналіз офіційних документів, прийнятих на конференції та доповідей на тему етики в науці, прочитаних на конференції. В результаті дебатів та обговорень учасників конференції було прийнято два важливих документи: 1) \foreignquote{ukrainian}{Декларація про науку та використання наукового знання} (Declaration on Science and the Use of Scientific Knowledge \cite[462]{WorldConferenceOnScience}) -- документ що підкреслює зобов'язання політиків щодо ведення наукової діяльності та вирішення проблем взаємодії між наукою та суспільством та 2) \foreignquote{ukrainian}{Науковий порядок денний – структура програми дій} (Science Agenda – Framework for Action \cite[476]{WorldConferenceOnScience}) -- документ що визначає принципи діяльності різноманітних категорій учасників конференції. Саме другий документ представляє особливий інтерес з точки зору етики в науці, бо містить загальноприйняті етичні принципи (Пункт 3.2, Етичні проблеми)\cite[481]{WorldConferenceOnScience}, якими повинні керуватися науковці. Оскільки цей документ був прийнятий після численних обговорень та консультацій із усіма членами ЮНЕСКО та науковими спільнотами цих країн, він може в широкому контексті слугувати як джерело визначення та пошуку способів вирішення етичних проблем, що виникають як наслідок ведення наукової діяльності \cite{Iaccarino2001-ju}. Через міжнародну важливість цього документи доцільним є підсумувати та проаналізувати основні тези принципів щодо етики, які декларує цей документ:

\begin{itemize}
    \item Етика та відповідальність повинна бути невід'ємною частиною підготовки усіх науковців. Особливу роль у моніторингу цього питання у майбутньому надається світовій комісії щодо питань етики наукового знання (COMEST) ЮНЕСКО разом з Постійний комітет Міжнародної наукової ради з питань відповідальності та етики наук (SCREWS). Тобто дуже важливим визначається потреба підготувати майбутніх науковців до моральних дилем з якими вони будуть стикатися в професійній діяльності, зробити максимум для заохочення усвідомлення, розуміння та рефлексії щодо моральних питань. Тобто шлях знайомства з етичними проблемами повинен розпочинатися обов'язково на етапі освіти науковця.
    \item Обов'язком наукових установ є сприяти вивченню етичних питань наукової роботи. Зазначено, що є необхідність в розробці спеціальних міждисциплінарних дослідницьких програм, щоб вивчати та моніторити етичні аспекти та наслідки та способи регулювання наукової роботи. Тобто в міжнародній науковій спільноті панує розуміння про необхідність подальших активних пошуків вирішення етичних проблем.
    \item Міжнародна наукова спільнота повинна співпрацювати з іншими, сприяти дебатам, включаючи публічну дискусію, пропагувати екологічну етику та екологічні кодекси поведінки. Отже акцентується увага на важливості залучення широкої публіки суспільства у дискусію про етичні питання та важливості етичних питань в галузі екології.
    \item Наукові установи закликаються дотримуватися етичних норм та поважати право свободи висловлювання науковців про етичні проблеми. Також наукові установи повинні засуджувати зловживання науковими та технологічними досягненнями та засуджувати їхнє неетичне використання. Отже важливим є право кожного науковця вільно висловлювати свою думку щодо етичних питань, адже лише таким чином дискусія зможе врахувати думку кожного.
    \item Урядові та неурядові організації мають організовувати дебати, включаючи публічні дебати, щодо етичних наслідків наукової роботи. Науковці, вчені та наукові організації повинні бути адекватно представлені в установах що ухвалюють рішення. Така діяльність повинна заохочуватися організаціями та повинна визнаватися як частина роботи та відповідальності науковця. Наукові асоціації повинні визначити етичний код для їхніх членів. У підсумку, цей пункт вказує на те, що у сучасному світі етичні питання та постійні обговорення цих питань (як всередині наукових кіл так і науковців з широким загалом) повинні бути невіддільними від наукової діяльності, тобто є інтегральною частиною наукової роботи.
    \item Уряди держав повинні заохочувати розробку релевантних механізмів для вирішення проблем пов'язаних з використанням та застосуванням наукового знання, а головне такі механізми повинні бути створені там де вони ще досі не існують. Неурядові організації та наукові комітети повинні сприяти утворенню комітетів з етичних питань в галузі їхньої компетенції.
\end{itemize}

Наведені вище тези стали міцною основою для розширення ролі етики в сучасній науці, етика в науці і надалі є об'єктом частих дебатів та дискусій як у наукових колах так і у громадянському суспільстві. 

\clearpage

\section[РОЗДІЛ 2. АКТУАЛЬНІ ПРОБЛЕМИ ЕТИКИ В НАУЦІ НА ПОЧАТКУ 21-ГО СТОЛІТТЯ]{АКТУАЛЬНІ ПРОБЛЕМИ ЕТИКИ В НАУЦІ НА ПОЧАТКУ 21-ГО СТОЛІТТЯ}

В даному розділі зупинимося на декількох конкретних моральних дилемах постмодерного інформаційного суспільства, щоб продемонструвати складність та багатогранність етичної сторони питання в науці. 

Розглянемо трохи вужче питання етики в галузі сучасних комп'ютерних технологій, яка є мені близькою. Помимо численних переваг та покращень у повсякденному житті, багато аспектів роботизації та впровадження цифрових технологій мають складні етичні аспекти. Немає сумнівів, що новітні цифрові технології роблять наше життя простішим та допомагають автоматизувати велику кількість операцій, але які ж наслідки мають впровадження таких технологій у морально-етичній площині?

Розглянемо питання етики при розробці і впровадженні в експлуатацію самокерованих автомобілів. Розробка таких автомобілів означає заміну водія, який наразі виконує складну процедуру управління автомобілем на цифрову комп'ютерну систему, яка здійснює ту саму процедуру на основі алгоритмів. На сьогоднішній день, прототипи самокерованих автомобілів, все частіше приймають участь в дорожньому русі і приблизно на 2050 рік планується розпочати їх комерційний продаж. В громадянському суспільстві вже десятиліття точаться дискусії про такі автомобілі, які підсилюються активним висвітленням в медіа. Також дискусії сприяють численні недавні аварії, спричинені або за участю самокерованих автомобілів \cite{holstein2018ethical}. 

Одним з найбільш привабливих аспектів самокерованих автомобілів, віра багатьох науковців в те, що вони можуть бути набагато безпечнішими за автомобілі керовані людиною \cite{nees2019safer}. Проте очевидно, що вони не можуть бути безпечними на 100\%, оскільки їздитимуть на високій швидкості в реальному середовищі, поруч з пішоходами, велосипедистами та іншими автомобілями. Отже постає питання, яким чином варто програмувати такі автомобілі, тобто як вони повинні реагувати у разі нещасного випадку і таке питання піднімає певні етичні проблеми. Скажімо, чи повинен автономний автомобіль намагатися врятувати максимальну кількість життів в абсолютному числі чи намагатися врятувати, в першу чергу, життя пасажирів? Які базові моральні принципи повинні бути закладені у такий \foreignquote{ukrainian}{алгоритм нещасного випадку}? Це питання має філософську природу і вже широко дискутується в медіа та в соціальних мережах. 

Науковці придумують конкретні приклади таких моральних дилем, та відзначають схожість таких прикладів на \foreignquote{ukrainian}{проблему з тролейбусом} \cite{thomson1986trolley}. Для того щоб розглянути і проаналізувати конкретний приклад, уявімо що автономний автомобіль повинен повернути наліво та зіткнутися з восьмирічною дівчинкою або ж повернути направо і збити вісімдесятирічну бабусю. Враховуючи швидкість авто, жертва точно загине, якщо не повернути, то загинуть обидві жертви, \cite{lin2016ethics}. Отже, в принципі, логічно було б повернути вправо чи вліво, щоб зберегти хоча б одне життя, проте яке саме рішення було би правильним з етичної сторони питання? Як би ви запрограмували таке рішення, не зважаючи на факт, що на практиці така ситуація буде ставатися надзвичайно рідко. На перший погляд, збити бабусю здається правильними рішенням, адже у дівчинки ще попереду усе життя, в той час як бабуся уже його прожила, тим більше маленька дитина буде морально невиннішою за будь-яку дорослу людину. З іншої сторони ми можемо вважати, що бабуся має таке ж право на життя що й дитина, навіть якщо вона погоджується пожертвувати своїм життям задля її порятунку. Але будь-який вибір в даному випадку не буде етично вірним. Багато організацій явно забороняють дискримінацію осіб за ознакою статі, раси, віку, релігії, гендеру, національності, сексуальної орієнтації, інвалідності тощо, наприклад Інститут інженерів з електротехніки та електроніки (IEEE), в своєму етичному кодексі \cite{ieee} декларує саме такі принципи. Дискримінація на основі віку у нашому експерименті виглядає як те саме зло що й дискримінація на основі раси, релігії, гендеру, інвалідності, національного походження і тому подібне, навіть якщо ми можемо придумати причини, щоб надати перевагу одній групі перед іншою. Більше того, такі принципи , що спираються на людське право на життя та гідність, можуть бути офіційно закріплені в конституціях країн в яких веде діяльність автовиробник, тому складно уявити як держава, в якій діє таке законодавство може дозволити компанії створити продукт, який може приймати подібні рішення. Якщо ж ми не можемо обрати що робити з етичної точки зору, то що нам залишається? Не приймати рішення взагалі і дозволити загинути обом жертвам? Це варіант, очевидно, здається значно гіршим ніж дати загинути лише одній жертві. Іншим способом вирішити цю проблему могло би бути прийняття випадкового рішення, не дискримінуючи жертву жодним чином, але і це рішення є проблемним з етичної точки зору, адже тоді виходить що ми виносимо вирок навіть не аналізуючи ситуацію. Спираючись даний поверхневий аналіз, бачимо, що ця проблема має багато валідних аргументів та немає простих рішень. А отже, це ще раз підтверджує гостру потребу в етичному аспекті в розробці автономних автомобілів. 

Наведемо ще один приклад моральної дилеми пов'язаної з вибором автономного автомобіля при потенційній аварії \cite{goodall2014ethical}. Уявімо, що перед автомобілем їдуть два мотоциклісти, один в шоломі, а другий без і ситуація складається таким чином, що автомобіль стикнеться або з одним або з другим і повинен прийняти рішення кого збити. ТУДУ мотоцикліст в шоломі

- cloning?
- self driving cars or AI?

\clearpage

\section*{ВИСНОВКИ}

Як бачимо, етика в науці стала інтегральною частиною наукової діяльності порівняно недавно. До цього спонукало все краще фінансування наукової роботи та збільшення співпраці академічних кіл та приватних організацій, орієнтованих на отримання прибутку. Загальними консенсусом світової наукової спільноти щодо етичного питання є принципи усвідомлення відповідальності науковця перед людством за свою діяльність та постійного оцінювання діяльності з точки зору етики. Дискусії науковців між собою та науковців з громадянським суспільством щодо етики в науці, а особливо щодо конкретних етичних проблем у різних сферах науки, активно продовжуються щодня. Саме такі дискусії та публічні дебати, на мою думку, допомагають вдосконалювати і переглядати погляди на етичність різних аспектів наукової діяльності.

Надзвичайно важливу роль у становленні сучасних принципів етики ведення наукової діяльності відіграв документ \foreignquote{ukrainian}{Науковий порядок денний – структура програми дій} (Science Agenda – Framework for Action \cite[476]{WorldConferenceOnScience}), що зафіксував після довготривалих дискусій загальний консенсус світової спільноти щодо питання етики в науці. На основі цього документу, коротко підсумовуючи принципи того, яке місце повинна займати етика в сучасній науці та які конкретні дії повинні здійснювати різні сторони причетні до науки, можемо виділити такі пункти: 1) етика повинна бути частиною освіти і навчання науковця 2) дослідницькі інституції повинні сприяти вивченню етичних питань наукової роботи 3) міжнародна наукова спільнота має пропагувати екологічну етику 4) наукові інституції повинні дотримуватися етичних норм. 5) уряди та громадянське суспільство повинні організовувати дебати про етичні наслідки наукової роботи. 6) уряди та громадянське суспільство повинне утворювати комітети з питань етики. 7) ЮНЕСКО має посилити свій Комітет з біоетики (Bioethics Committee) та Всесвітню комісію з етики наукових знань (World Commission on the Ethics of Scientific Knowledge). Саме ці принципи, які водночас є і закликом до дії, можемо на сьогоднішній день вважати актуальними принципами існування етики в сучасній науці, оскільки вони були сформульовані після численних дискусій світової наукової спільноти. Основи закладені цим документом були розвинуті, доповнені та адаптовані до сучасних умов подальшими деклараціями Світового форуму науки \cite{WorldScienceForumDeclaration2019}, який на сьогодні відіграє роль майданчика для дискусій міжнародної наукової спільноти щодо етичних питань.

Центральним питанням етики в науці є обговорення моральних дилем, тобто питань на які немає чіткої відповіді з етичної точки зору. На прикладі самокерованих автомобілів, бачимо складність та багатогранність таких проблем, та способи за допомогою яких люди намагаються підходити до їх вирішення. Хоча етичних проблем з сучасними технологіями та розробками виникає все більше, я певен, що людство зможе розробити ефективну методологію вирішення таких проблем.

\addcontentsline{toc}{section}{ВИСНОВКИ}