\begin{center}
{\large \textbf{АНОТАЦІЯ}}
\end{center}

\textbf{Леськів Т. С. \thesistopic. --– Рукопис.}  \\
Дисертація на здобуття наукового ступеню доктора наук з державного управління за спеціальністю 25.00.02 --– механізми державного управління – Національний університет \foreignquote{ukrainian}{Львівська політехніка}. --– Львів, 2026.

\textbf{Ключові слова}: цифровізація публічного управління, цифрова трансформація, е-урядування, цифрові сервіси.

\bigskip

\clearpage

\begin{center}
{\large \textbf{ANNOTATION}}
\end{center}

\textbf{Leskiv T.S.} Optimization of luminosity measurements for the LHCb experiment. \\
Masters qualification work in specialty 104 Physics and astronomy, educational program \foreignquote{ukrainian}{High energy physics}. --- Lviv Polytechnic National University. --- Lviv, 2026.

\textbf{Research supervisor}: HDR Sergey Barsuk, Laboratoire de Physique des 2 Infinis Ir\`{e}ne Joliot-Curie (IJCLab), Orsay, France

\textbf{Сurator from the department}: PhD of Physics and Mathematics, Assoc. Prof. Bezshyyko O., Associate Professor of Department of Nuclear Physics.

\textbf{Key words}: digitalization of public administration, digital transformation, e-government, digital services.

Translation of the abstract.
