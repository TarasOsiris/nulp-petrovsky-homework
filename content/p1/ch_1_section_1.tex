Цифрові інформаційно-комунікаційні технології з кожним днем усе більше розмивають межі між фізичною, цифровою та біологічною сферами і в прискореному темпі трансформують людське життя, роботу та комунікації. Зокрема, й у публічному секторі, з точки зору законодавства, інституцій, стратегій та інструментів усе більше розмивається межа між урядуванням та цифровим урядуванням (або електронним урядуванням). Саме цей процес проникнення цифрових технологій в усі сфери урядування, який суттєво активізувався в останні десятиліття, можемо називати трендом цифрової трансформації публічного управління та впровадження електронного урядування, що є частиною загальносвітового тренду проникнення цифровізації у всі сфери соціальної діяльності людини. З розвитком електронного урядування, публічні адміністрації та інституції у всьому світі зазнають незворотних перетворень, водночас структурних та функціональних з точки зору динаміки взаємодії між урядами держав та людьми, яким вони служать \cite[22]{book:un_e_govt_survey_2022}. Саме такі загальні спостереження, що базуються на більш ніж двадцятирічних аналітичних дослідженнях та моніторингу трендів розвитку цифрового публічного управління, фіксує останнє видання (на момент написання цієї роботи) «Дослідження е-урядування ООН» 2022 року, що є найбільш повним та масштабним дослідженням прогресу та розвитку впровадження е-урядування в світі та охоплює 193 країни члени ООН. Аналіз підсумків останнього видання також свідчить про продовження закономірного здійснення поступового переходу від традиційного технократичного підходу е-урядування до впровадження стратегій цифрового розвитку, які є орієнтовані на політику та дані і є політично керованими, а також про те, що у сучасному інформаційному постмодерному суспільстві цифровізація перевизначає та трансформує роботу урядів. 

Також з настанням пандемії COVID-19, цифрові технології відіграли ключову роль в боротьбі з її наслідками, уможливлюючи надання основних державних послуг та базових послуг в галузях охорони здоров'я, освіти та безпеки в той час як фізичний доступ до цих послуг став все більше обмеженим. Пандемія ще більше підсилила важливість е-урядування та цифрових технологій як ключових інструментів комунікацій між урядом, приватним сектором та суспільствами у всьому світі. Рух в сторону цифрової трансформації став незворотнім, оскільки цифрові технології роблять важливий внесок як в державний так і в регіональний розвиток, уможливлюють поширення знань та практик, та дають можливість надавати послуги та рішення користувачам як і при нормальних так і при надзвичайних обставинах \cite[32]{book:un_e_govt_survey_2022}.

Сьогодні впровадження сучасних інформаційно-комунікаційних технологій (ІКТ) у різноманітні сфери суспільного життя та діяльність органів влади усіх рівнів ієрархії управління є одним із основних напрямів розвитку державної політики в Україні. Для оцінки успішності розвитку е-урядування в Україні з іншими державами можемо відзначити приналежність до групи держав з дуже високим значенням EGDI 2022 року -- VHEGDI, клас рейтингу V1 \cite{online:egdi}. Порівнюючи з показниками 2020 року, Україна перемістилася з групи з високим EGDI (0.7119, клас рейтингу HV) в групу з дуже високим EGDI (0.8029, клас рейтингу V1), та перемістилася вверх на 23 сходинки з 69-го на 46-те місце в рейтингу держав-членів ООН, що, в загальному, свідчить про значний ривок у розвитку е-урядування в державі в цей період. \cite[36]{book:un_e_govt_survey_2022}. Не зважаючи на те, що є країною з доходами нижче середнього, Україна має дуже високий OSI (0.8148), що свідчить про доволі високий рівень розвитку національних онлайн-сервісів \cite[44]{book:un_e_govt_survey_2022}\cite{online:egdi}.\textit{[TODO про процес в Україні коротко, стратегія і т.д.]}\textit{[Complex network analysis position \cite[47]{book:un_e_govt_survey_2022}]}

\textit{TODO зробити таблиці на основі, подати недоліки - from https://publicadministration.un.org/egovkb/Data-Center, https://publicadministration.un.org/egovkb/en-us/Data/Country-Information/id/180-Ukraine/dataYear/2022}

Для ефективного впровадження процесу цифрової трансформації публічного управління та розвитку е-урядування, як і у будь-якій іншій сфері державного управління, необхідним є використання грошей, матеріально-технічних засобів, кадрового персоналу та інших різноманітних ресурсів. Для позначення сукупності усіх таких ресурсів, що є необхідними або можуть бути використані для цього процесу, доцільним є впровадження поняття \foreignquote{ukrainian}{ресурсів цифрової трансформації урядування}. Для того щоб сформулювати визначення поняття \foreignquote{ukrainian}{ресурс цифрової трансформації урядування} доцільно спочатку розглянути сучасну методологію та підходи до дослідження та аналізу таких явищ як \foreignquote{ukrainian}{е-урядування} та \foreignquote{ukrainian}{цифровізація державного управління}, актуальні загальносвітові тренди їх розвитку та ключові цінності, які лежать в основі створення стратегій та тактик їх розвитку і реалізацію таких стратегій і тактик за допомогою інституційних реформ та інших інструментів.

Методологічна основа для збору та обробки даних \foreignquote{ukrainian}{Дослідження е-урядування ООН} \cite{book:un_e_govt_survey_2022}\cite{book:un_e_govt_survey_2020}, що з 2001 року досліджує прогрес і ефективність цифровізації державного управління держав членів ООН, базується на цілісному баченні е-урядування та включає в себе такі три основні виміри, які дозволяють людям отримувати вигоду від онлайн-сервісів та інформації онлайн: 1) достатність і відповідність актуальним потребам телекомунікаційної інфраструктури 2) здатність людських ресурсів просувати та використовувати ІКТ 3) наявність та доступність онлайн-сервісів та ресурсів. Результатом даного дослідження є виведення Індексу розвитку е-урядування ООН (United Nations E-Government Development Index(EGDI)) для кожної окремої держави, який є композитним індексом та складається в свою чергу з нормалізованих Індексу телекомунікаційної інфраструктури (Telecommunications Infrastructure Index (TII)), Індексу людського капіталу (Human Capital Index (HCI)) та Індексу онлайн сервісів (Online Service Index (OSI)) \cite[21]{book:un_e_govt_survey_2020}. Також дане дослідження визначає такі дев'ять ключових основ цифрової трансформації публічного управління \cite[22]{book:un_e_govt_survey_2020}: 

% TODO - зробити схему рисунок
\begin{enumerate}
    \item \textit{Бачення (візіонерство), лідерство та тип мислення в напрямі цифрової трансформації публічного управління.} Полягає у потребі зміцнення лідерства орієнтованого на провадження цифрової трансформації та у зміні способу мислення та підвищенні цифрових можливостей та цифрової грамотності конкретного індивіда. Під цифровою грамотністю тут розуміємо наявність навичок та вміння особи користуватися різноманітними цифровими пристроями та сервісами.
    \item \textit{Інституційна та нормативна база.} Полягає у необхідності розробки та розвитку інтегрованої інституційної екосистеми за допомогою комплексної нормативно-правової бази.
    \item \textit{Організаційна структура та культура.} Полягає у трансформації структури та культури організації таким чином, щоб забезпечувати максимальну ефективність електронного урядування.
    \item \textit{Системне мислення та інтеграція.} Полягає у просуванні системного мислення та розвитку інтегрованих підходів до формування політики та доставки цифрових сервісів.
    \item \textit{Управління даними.} Полягає в забезпеченні стратегічного та професійного менеджменту даних, для уможливлення розробки політики на основі даних та доступу до інформаційних ресурсів через відкриті джерела урядових даних, окрім інших пріоритетів доступу та використання даних.
    \item \textit{Інформаційно-комунікаційна інфраструктура (ІКТ).} Полягає у забезпеченні фінансової та фізичної доступності до неї, зокрема доступу до мережі Інтернет та засобів доступу до неї, тобто комп'ютерів, мобільних пристроїв зв'язку тощо.
    \item \textit{Фінансові ресурси.} Полягає у мобілізації фінансових та інших ресурсів та узгодженні пріоритетів, планів та бюджетування, включаючи також їхню реалізацію через державно-приватне партнерство.
    \item \textit{Спроможність розробників потенціалу цифрової трансформації уряду.} Полягає у підвищенні спроможностей та компетенції освітніх закладів, що займаються підготовкою державних службовців та інших інституцій.
    \item \textit{Суспільні можливості.} Полягає у розвитку потенціалу на суспільному рівні, щоб нікого не залишити позаду і подолати цифровий розрив. Зокрема, полягає у вирішенні проблеми, що групи, до яких найлегше дістатися, переважно отримують найбільшу кількість переваг від суттєвого розвитку електронного урядування, в той самий час коли найбідніше та найвразливіше населення були залишені позаду. Протягом останніх двадцяти років здійснювалися важливі просування в електронному урядування, проте недостатня увага приділялася інклюзивному дизайну \cite[28]{book:un_e_govt_survey_2022}.
\end{enumerate}

Таким чином, ресурси цифрової трансформації урядування -- це сукупність усіх наявних видів ресурсів (фінансових, інформаційних, соціальних, адміністративних, матеріально-технічних, інфраструктурних, інтелектуальних, кадрових тощо), що використовуються або потенційно можуть бути ефективно використаними суб'єктами управління з метою досягнення поставлених цілей при здійсненні діяльності впровадження цифрової трансформації державного управління та розвитку е-урядування або з метою покращення ефективності та вдосконалення такої діяльності.

Різноманіття конкретних ресурсів цифрової трансформації, які використовуються в державному управлінні, породжує та актуалізує проблему їхньої класифікації. Оскільки конкретних таких ресурсів є велика кількість, проблема класифікації полягає у виборі таких критеріїв, базуючись на яких можливо звести відомі ресурси цифрової трансформації до однорідних груп зі схожими властивостями.

\textit{TODO Тут подаємо опис проблеми, а саме брак цифрових ресурсів в громадах.}

Одним із видів ресурсів цифрової трансформації публічного урядування є цифрові онлайн-сервіси. \textit{TODO - Ukraine EGDI, Online services index.}